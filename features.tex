\documentclass[index]{subfiles}
\begin{document}
\chapter{无线充电系统优化问题及其特点}\label{sec:fea}
本章将首先给出一般性电动汽车无线充电系统磁谐振部件优化设计问题的形式化描述,
再逐一分析这一问题的特征,
并在此基础上概述解决问题的大致思路和注意事项。
后续章节将会对解决问题的具体手段进行详述。

\section{问题描述}
根据\cref{sec:intro}中的讨论,本文所研究的问题的具体表述如下:

\begin{definition}[一般性电动汽车无线充电系统磁谐振部件优化设计问题]
  寻找一组或多组设计变量的组合,使得电动汽车无线充电系统的总体性能指标达到(在所有可能设计变量组合中的)最优;
  其中性能指标与设计变量的关系为:
  先从设计变量构建磁耦合元件的几何模型和磁模型,
  再对某一个或者多个工况下的磁模型进行有限元分析得到电模型,
  最后从上述模型中计算得到性能参数。
\end{definition}

\begin{definition}[设计变量]
  分为以下几种:
  \begin{description}
    \item[分类型设计变量] 可以在某个有限集合中取值,集合中的元素互相没有任何关系。
      常见的分类型设计变量包括线圈形状(圆形、方形、DD形等)、磁芯材料等等。
    \item[离散型设计变量] 可以在某个有限闭区间内取任意整数值。
    常见的离散型设计变量包括线圈匝数、线圈层数等等。
    \item[连续型设计变量] 可以在某个有限闭区间内取任意实数值,但无需区分距离小于某一阈值的两个值。
      常见的连续型设计变量包括线圈尺寸、磁芯尺寸、磁芯位置等等。
      之所以无需对小于某一阈值的两个值进行区分,是因为在工程上对半径为\SI{10}{\centi\meter}和\SI{10.0001}{\centi\meter}的线圈进行区分没有实际意义。
  \end{description}
\end{definition}
问题 设计变量的类型
特点 样本空间离散 运行时间长 外部依赖项多
手段 
注意事项

% TODO
\section{确定性}
% 同一设计只有一个结果
% TODO
\section{并行性}
% 同时执行多个仿真
% TODO
\section{重复性}
% 手工执行仿真
% TODO
\section{离散性}\label{sec:fea-discrete}
% 无限精度没有意义
% TODO
\section{相关性}
% 设计相近效果相近
% TODO
\end{document}
