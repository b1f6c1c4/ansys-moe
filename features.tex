\documentclass[index]{subfiles}
\begin{document}
\chapter{无线充电系统优化问题及其特点}\label{sec:fea}
本章将首先给出一般性电动汽车无线充电系统磁谐振部件优化设计问题的形式化描述,
再逐一分析这一问题的特征,
并在此基础上概述解决问题的大致思路和注意事项。
后续章节将会对解决问题的具体手段进行详述。

\section{问题描述}
为了明确本文讨论问题的边界,也为了给后续分析打下记录,
根据\cref{sec:intro}中的讨论,本节将先给出本文所研究的问题的具体表述,在对各个细节分别给出详细的定义,并辅以例子进行阐释。

\begin{definition}[一般性电动汽车无线充电系统磁谐振部件单目标优化设计问题]\label{def:problem}
  寻找一组或多组设计变量的组合,使得电动汽车无线充电系统的总体性能指标(目标函数)达到(在所有可能设计变量组合中的)最优;
  其中总体性能指标与设计变量的关系为:
  先从设计变量构建磁耦合元件的几何模型和磁模型,
  再对某一个或者多个工况下的磁模型进行解算(一般是有限元分析)得到电模型,
  进一步从上述模型中计算得到性能指标,
  最后将各个性能指标进行线性或者非线性组合以得到总体性能指标。
\end{definition}

\subsection{设计变量}
\begin{definition}[设计变量]\label{def:dvars}
  设计变量分为以下几种:
  \begin{description}
    \item[分类型设计变量] 可以在某个有限集合中取值,集合中的元素互相没有任何关系。
    \item[离散型设计变量] 可以在某个有限闭区间内取任意整数值。
    \item[连续型设计变量] 可以在某个有限闭区间内取任意实数值,但无需区分距离小于某一阈值的两个值。
  \end{description}
  一个设计变量不一定在任何情况下都有意义(需要设计)。
  一个设计变量是否有意义可以依赖于其它分类变量的取值(但不能成环)。
\end{definition}
\begin{remark}[连续型设计变量的阈值]
  之所以无需对小于某一阈值的两个值进行区分,是因为在工程上对半径为\SI{10}{\centi\meter}和\SI{10.0001}{\centi\meter}的线圈进行区分没有实际意义。
\end{remark}
\begin{example}[设计变量的分类]
  常见的分类型设计变量包括线圈形状(圆形、方形、DD形等)、磁芯材料等等。
  常见的离散型设计变量包括线圈匝数、线圈层数等等。
  常见的连续型设计变量包括线圈尺寸、磁芯尺寸、磁芯位置等等。
\end{example}
\begin{example}[设计变量的依赖关系]
  用一个分类变量dShape表示线圈形状,
  而分别用连续型设计变量dRadius表示圆形线圈的半径(只在dShape为圆形时有意义),用连续型设计变量dLength示方形线圈的长度(只在dShape为方形时有意义)。
\end{example}

\subsection{几何模型}
\begin{definition}[几何模型]
  用于表示系统磁耦合元件的几何特征,包括若干个参数。每个参数都是实数,由设计变量经过计算得到。各个参数可以互相依赖(但不能成环)。
  计算过程既可以是简单的算术表达式,也可以交由外部程序(如Python、R语言、Mathematica等)完成。
\end{definition}
\begin{example}[几何模型的参数]
  对于环形线圈的设计,若设计变量dTurns表示匝数,dInt表示匝间距,dMinRadius表示内径,则用参数gMaxRadius表示几何模型的外径,
  可以通过算术表达式$\textrm{gMaxRadius}=\textrm{dTurns}\times\textrm{dInt}+\textrm{dMinRadius}$计算得到。
\end{example}

\subsection{磁模型}
\begin{definition}[磁模型]
  用于表示系统磁耦合元件的磁路特征,一般需要通过对几何模型进行有限元分析得到。
  在有限元分析软件中,导入若干几何参数,指定磁模型设置,运行仿真,再将所需的磁路参数导出即可。
\end{definition}
\begin{remark}[不采用有限元分析计算磁模型]
  对于采用解析法求解几何模型得到磁模型的优化问题,可以将对解析法计算程序的调用视作对一个新的几何模型(或者电模型)的参数的调用,并忽略有限元分析的步骤。
\end{remark}

\subsection{电模型}
\begin{definition}[电模型]
  用于表示无线充电系统的电路特征,包括若干个参数。每个参数都是实数,由设计变量、几何模型参数、磁模型参数经过计算得到。各个参数可以互相依赖(但不能成环)。
  计算过程既可以是简单的算术表达式,也可以交由外部程序(如Python、R语言、Mathematica等)完成。
\end{definition}
\begin{example}[电模型的参数]
  设磁模型参数mResistance表示发射线圈的电阻(由有限元分析得到),mInductance表示发射线圈的自感(由有限元分析得到),则用参数eImpedance表示电模型中的发射线圈感抗的模值,
  可以通过算术表达式$\textrm{eImpedance}=\sqrt{\textrm{mResistance}^2+\textrm{mInductance}^2}$计算得到。
\end{example}

\subsection{性能指标}
\begin{definition}[性能指标]
  用于表示无线充电系统的单项性能,包括若干个指标。每个指标都是实数,由设计变量、几何模型参数、磁模型参数、电模型参数经过计算得到。各个参数可以互相依赖(但不能成环)。
  计算过程既可以是简单的算术表达式,也可以交由外部程序(如Python、R语言、Mathematica等)完成。
\end{definition}
\begin{definition}[总体性能指标]
  用于表示无线充电系统的总体性能,是一个实数,由设计变量、几何模型参数、磁模型参数、电模型参数、各性能指标经过计算得到。
  计算过程只能是简单的算术表达式。
\end{definition}

\section{问题特点}\label{sec:fea-fea}
根据上述对优化设计问题的详细定义,不难得出该问题具有如下特点:
\begin{description}
  \item[确定性] 给定一组设计变量,所有模型的所有参数均已确定
  \item[并行性] 不同组设计变量的模型之间的计算互不干扰,可以并行执行
  \item[重复性] 对不同设计变量计算各种模型的参数的工作属于机械性的简单重复劳动
  \item[离散性] 有意义的设计变量的组合数目是有限的(即便连续型设计变量,也因为其精度有限而只包括有限种不同的情况)
  \item[相关性] 对离散型和连续型设计变量,两组设计在取值相近的情况下,各模型参数也基本相近
  \item[昂贵性] 已知一组设计变量,求取目标函数值的过程需要耗费大量的时间
  \item[外部依赖性] 不论使用何种优化算法,都需要与外部程序进行数据交互才能进行
    (调用有限元分析程序求解磁模型,调用Python、R语言、Mathematica等程序语言和数学软件进行复杂参数的计算)
\end{description}

\section{解决思路}
% TODO
\end{document}
