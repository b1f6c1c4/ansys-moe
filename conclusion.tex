\documentclass[index]{subfiles}
\begin{document}
\chapter{结论}\label{sec:con}
现有文献在电动汽车无线充电系统磁谐振部件的设计问题上存在着严重不足:
要么没能同时对多个设计变量进行性能调优,要么对优化方法的选择并不十分恰当。
大部分文献为了简单起见选择了对每个设计变量依次优化,放弃了很多设计可能性;
其余文献选择了暴力搜索、启发式算法全局最优化算法(参见\cref{sec:intro-opt}),然而这些算法并不适合电动汽车无线充电系统磁谐振部件设计问题(参见\cref{sec:go}的分析)。
另外,现有文献对优化设计问题本身并没有一般化的讨论,而大都以具体问题的具体设计展开讨论,参考意义不大。

本文针对以上不足,首先形式化地研究了电动汽车无线充电系统磁谐振部件优化设计问题本身的结构和特点,
再针对此展开文献研究,找到了最适合于该问题的贝叶斯优化算法,
并利用软件工程的手段将该算法完整地实现了出来。
针对有一定挑战性(5个独立设计变量,6个设计目标,设计可能性\num{4.76e6}种,每种可能性计算耗时\SI{30}{\minute},暴力搜索耗时\num{272}年)
的具体优化设计算例的运行结果显示,仅使用1台6核工作站无人值守运行\SI{21}{\hour},便得到了53组满足设计全部设计要求的方案。

\end{document}
