\documentclass[index]{subfiles}
\begin{document}
\chapter{结论}\label{sec:con}
现有文献在电动汽车无线充电系统磁谐振部件的设计问题上存在着严重不足:
要么没能同时对多个设计变量进行性能调优,要么对优化方法的选择并不十分恰当。
大部分文献为了简单起见选择了对每个设计变量依次优化,放弃了很多设计可能性;
其余文献选择了暴力搜索、启发式算法全局最优化算法(参见\cref{sec:intro-opt}),然而这些算法并不适合电动汽车无线充电系统磁谐振部件设计问题(参见\cref{sec:go}的分析)。
另外,现有文献对WPT磁谐振部件的优化设计问题本身并没有一般化的讨论,而大都以具体问题的具体设计展开讨论,参考意义不大。

本文针对以上不足,首先定性地研究了电动汽车无线充电系统磁谐振部件优化设计问题本身的结构和特点,
再针对此展开文献研究,找到了最适合于该问题的贝叶斯优化算法,
并利用软件工程的手段将该算法完整地实现了出来。
针对有一定挑战性(5个独立设计变量,6个设计目标,设计可能性\num{4.76e6}种,每种可能性计算耗时\SI{30}{\minute},单核暴力搜索将耗时\num{272}年)
的具体优化设计算例的运行结果显示,仅使用1台6核工作站无人值守运行\exmHour{},便得到了\exmFea{}组满足设计全部设计要求的方案。
从实际搜索位置的分布情况来看,该算法在搜索位置上的选取较为合理,比起暴力搜索和完全随机搜索都节约了巨量的计算资源;
从系统本身来看,该系统对可以实现无人值守自动运行,出现错误以后能将问题控制在最小范围内,并在问题解决之后从上次中断位置无缝继续执行。

必须指出的是,该系统在算法上和实现上还存在一定的不足。在算法理论上,本文还存在以下不足:
\begin{description}
  \item[对不同分类之间的联系没有加以考量] 若某一分类中目标函数普遍在$-100$左右,而另一分类的目标函数普遍只有$-10$的水平,
  则应该将全部算力用于更有前途的前者上,而非两者平等地共享计算资源;这可以通过分别求解每个分类的EPI,并选取其中最高者来实现。
  \item[无法真正处理多目标情况] 贝叶斯优化算法只能处理单目标的优化问题,而电动汽车无线充电系统磁谐振部件的设计问题是典型的多目标问题。
  本文所采用的折衷办法是将各目标函数线性组合起来作为优化目标,虽然能够勉强解决问题但还有很大的改良空间;
  若将EI的定义扩展成某一位置采样时对帕累托前沿的提升的期望,有可能对解决该问题提供帮助。
  \item[无法真正处理含有约束的情况] 有约束的优化问题的贝叶斯优化算法是一个疑难问题,目前还没有得到广泛认可的解决方案。
  本文所采用的折衷办法即将超出约束的位置的目标函数直接设置成0(或者其他固定数值),这样虽然最终结果是正确,但有可能因为频繁出界而大大延长计算时间;
  有可能的解决办法是在求解优化子问题时就对约束进行考量,或者妥善选取设计变量使得新的设计变量只需遵从闭区间一组约束条件。
\end{description}

另一方面,在工程实际方面,本文也远非完美:
\begin{description}
  \item[运行开始后无法调整设计变量] 如果自动化设计系统能够结合人类工程师的智慧,在优化运行期间动态扩大或者缩小设计变量取值范围,
  那么就能将更多算力集中在更有可能的设计空间区域内,大大加速收敛。
  \item[出现错误以后只能由管理员恢复] 由于系统交互结构复杂,最终用户难以对错误进行定位,也无法(没有对用户暴露相应端口)手动修改状态以修复错误,
  相关工作只能由对系统结构熟悉的管理员来进行;可以通过增加更多的管理控制接口来缓解该问题。
  \item[可缩放性一般] 对Petri网的模拟采用了单核单线程机制,使得在需要同时处理很多(根据系统性能,可能是几千或者几万)个独立的优化问题时性能有可能遇到瓶颈;
  可以通过根据项目不同来将负荷分摊在多个并行执行的系统内核上。
  \item[无法处理迭代次数很多的情况] 随着迭代数目增多,高斯过程参数估计、子优化问题的求解所消耗的时间也快速增加(仅矩阵乘法就需要$\Omega(n^3)$的时间):
  对于已经进行了250次迭代的子问题,在实际系统上实测需要消耗超过\SI{10}{\minute}的时间来进行高斯过程参数估计。
  不难想象如果“判断应该在哪里采样”的时间相当于甚至超过了“在某个位置采样”所实际消耗的时间时,贝叶斯优化就失去了它的意义。
  为此,一方面需要降低高斯过程优化估计问题的复杂性(如采用更简单更快速而非最准确的方式),另一方面可以适时发布非最优位置的计算任务,
  宁可在较差位置进行一次采样,也要避免计算资源空闲浪费。
\end{description}

\end{document}
