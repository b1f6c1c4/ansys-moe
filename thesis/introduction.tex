\documentclass[index]{subfiles}
\begin{document}
\chapter{引言}\label{sec:intro}
\section{研究背景}
无线电能传输(Wireless Power Transmission)技术最早可以追溯到19世纪末尼古拉·特斯拉时期的系列研究\cite{barrett1894}。
这一技术对交通领域带来了一定的影响,有不少将无线电能传输技术应用在交通领域的初步的尝试\cite{eghtesadi1990}。
在2007年补偿网络方法由Solja\v{c}i\'c引入\cite{kurs2007}之后,基于补偿网络的磁谐振耦合式无线充电技术迅猛发展,
并以其传输距离长、传输功率大、传输效率高、成本低廉等诸多优势,在电动汽车领域取得了广泛的应用\cite{高大威2015}。

\begin{figure}[htb]
  \centering
  \includegraphics{./figures/dist/intro-arch.pdf}
  \caption[电动汽车无线充电系统技术体系]{电动汽车无线充电系统技术体系。节选并改编自文献~\inlinecite{高大威2015}。
  加粗带下划线部分为本文所重点研究的部分。\label{fig:intro-arch}}
\end{figure}

对磁谐振耦合式电动汽车无线充电系统来说,它是一个多物理系统耦合的复杂整体,与电力电子、电磁场等诸多学科互相交叉,需要大量的相关技术进行支撑。
文献~\inlinecite{高大威2015}从系统的本质属性和基本要求出发,讨论了电动汽车无线充电系统中存在的科学问题和相应的支撑技术。
为了提高系统效率,需要解决多源能量双向耦合管理的科学问题;其中关键的一项也就是对磁谐振部件(即传输线圈、铁芯等)进行电磁场分析、计算与仿真。
实际上,仅对电磁场进行仿真解算还远远不够,应该利用计算结果对这些部件的材料、形状、尺寸等参数进行合理地调整,以得到更高的效率。
这一过程一般称作优化设计。基于电磁场仿真的优化设计过程牵扯到有限元方法、计算机实验设计、全局最优化等数学工具,
且往往需要采用软件工程的方法,开发自动化设计软件来辅助设计者完成系统的优化设计。
\cref{fig:intro-arch}从电动汽车无线充电系统的本质属性和基本要求出发,完整交代了本文的选题在上述知识体系(只展开描绘了和本文有直接联系的部分)中的位置,
为后续分析描绘了基本图景。

\section{研究现状}
对于无线充电系统磁谐振部件的设计问题,现有文献已经给出了相当多的解决方案。
综述~\inlinecite{张艺明2016}总结了几种适用于通用磁谐振式WPT系统的传输线圈和铁芯的结构。
而针对电动汽车这一特殊应用,磁谐振部件可以根据充电过程中汽车的运动状态分为动态与静态两大类\cite{赵争鸣2016}。
由于动态充电系统(发射端一般是通电轨道,接收端一般是汽车上的接收线圈)的物理结构和设计分析方法都相对复杂,本文将讨论局限在静态(也即固定式)充电系统。

\subsection{线圈结构}
固定式电动汽车无线充电系统的磁耦合系统按基本结构划分,可以分为平面式(亦称单边式、平板式)和螺线管式(亦称双边式、圆柱式)\cite{高大威2015}。
平面式的基本特点是(对发射端或者接收端而言)铁磁材料位于下方,传输线圈水平放在其上;
而螺线管式的基本特点则是铁磁材料位于中央,传输线圈缠绕在铁磁材料之上。

平面式磁耦合系统再做细分可以分为圆形\cite{budhia2011}、DD\cite{budhia2013}、DDQ\cite{budhia2013}、BP\cite{covic2011}、方形\cite{choi2014}等等。
文献~\inlinecite{zaheer2012,zaheer2015}对比了上述几种线圈结构并指出采用DDQ和BP作为接收线圈具有较好的充电效果。

关于螺线管式磁耦合系统,常见结构包括Flux pipe\cite{budhia2010}、H\cite{takanashi2012}等。
文献~\inlinecite{haldi2013}利用有限元分析的手段,对平面式磁耦合系统和螺线管式磁耦合系统进行了详细的比较,并指出平面式磁耦合系统相比于螺线管式有着更多的优势。
但也有文献~\inlinecite{唐云宇2015}指出平面式结构存在诸多缺点,如有零耦合偏移点、磁通分布路径半径小等等,并建议采用螺线管式结构。
关于两种结构的讨论还远未得出最终结论,仍然需要更多的研究。

\subsection{优化设计}\label{sec:intro-opt}
许多文献或多或少对磁谐振优化设计问题进行了研究,也即调节线圈和铁芯的形状、材料、尺寸等参数以使得系统性能达到相对最佳。
目前已知大部分文献采用一次搜索一个变量的方法来进行优化设计:
文献~\inlinecite{郑广君2017}依次对利兹线、线圈形状、线圈串联结构、长宽比、工作频率等参数进行设计。
文献~\inlinecite{zhao2017}分别对线圈和铁芯的尺寸进行设计。
文献~\inlinecite{ye2016}采用解析方法对线圈匝数进行优化设计。
文献~\inlinecite{tang2015}依次讨论了匝间距、线圈长度、磁芯位置、屏蔽结构的选择。
文献~\inlinecite{shijo2015}独立地研究了铁芯形状和铁芯的厚度。
不难看出,这些方法虽然能起到一定的性能调优的目的,但距离在所有的设计可能性中找到一个最佳方案还有着很大的空间。

文献~\inlinecite{movagharnejad2016}给出了一个通用的电动汽车无线充电磁谐振系统的优化设计工作流,依次选择各个参数,再检验是否满足设计要求。
遗憾的是,文中对该工作流中性能参数的具体计算和选择语焉不详,更没有讨论在设计不满足要求时该如何进行调整。

文献~\inlinecite{deng2016}尝试同时对线圈尺寸、补偿电容、谐振频率三者进行优化。
然而,该文采用了在可行域内沿某一固定方向按给定步长进行线性搜索的方式来进行计算,显然不能找到全局最优设计。

若要找到全局最优设计,则要么遍历所有可能的设计组合,要么就采用全局最优化算法。
文献~\inlinecite{yilmaz2017}真正从全局最优化算法的角度研究了如何同时对多个设计变量进行优化:
作者同时考虑了线圈位置和形状,建立设计变量的离散表示,利用Matlab实现了多目标混合粒子群优化算法(MOHPSO),
再从Matlab调用Ansys仿真程序以进行具体电磁场解算。此文同时考虑了多种可能的设计目标组合,并充分考虑不同设计目标之间权衡情况,给出了多组帕累托前沿(Pareto Front)。
文献~\inlinecite{bandyopadhyay2016,bandyopadhyay2017}研究了类似问题,只不过采用了不同的设计变量优化目标组合。
同样基于全局最优化算法的类似文献还包括~\inlinecite{ning2014,han2014},只不过使用了遗传算法(GA)而非粒子群优化算法(PSO)。

然而,这一方面的研究还有相当大的空缺。
一方面,一些更为优秀的全局最优化方法,如已经应用到了光伏发电最大功率点追踪\cite{abdelrahman2016}的贝叶斯优化算法,
还没有被引入电动汽车无线充电系统磁谐振部件的优化设计方法中;
另一方面,上述所有文献都只是就事论事地讨论某一个具体设计问题下如何进行设计决策,
而对于一般性的设计问题既没有一个很好的概括,也没有相应的软件实现。

\section{论文结构}
本文针对以上这些不足之处,在充分考虑一般性电动汽车无线充电系统磁谐振部件优化设计问题的特点(\cref{sec:fea})的基础上,
首先对现代全局最优化方法进行系统化的回顾(\cref{sec:doebgo}),找到适用于该问题的优化算法——贝叶斯优化算法(详见\cref{sec:bgo}),
再展开相应的软件研究——设计、开发一套自动化设计系统(\cref{sec:petri,sec:design,sec:impl}),最后利用具体算例验证该系统的有效性(\cref{sec:example})。

由于设计问题的复杂性,自动化设计系统的软件结构异常复杂;为此,本文先对现有的软件结构建模工具进行扩展(\cref{sec:petri}),再依据软件工程的原则,按自顶向下的顺序,先建模系统顶层架构(\cref{sec:design}),再讨论系统底层实现(\cref{sec:impl})。
最后,在\cref{sec:example}中,将会通过一个实际无线充电系统的优化算例来对系统的有效性进行验证。

\end{document}
