\documentclass[index]{subfiles}
\begin{document}
\chapter{从计算机实验设计到贝叶斯优化}
计算机实验设计\footnote{需要注意的是,本文中术语“计算机实验设计”指的是Design of Computer Experiments,与“计算机辅助的实验设计”(Computer-Aided Design of Experiments)有明显的不同——前者的实验对象仅限计算机程序,后者对实验对象没有限制。}是20世纪末随着电子计算机的蓬勃发展而产生的新兴学科\cite{mckay1979}。
计算机实验设计方法和思想的进展扩展了全局最优化问题的求解思路。
本章将先分别回顾计算机实验设计和全局最优化两个学科独立发展的历史,
再着重介绍两者的有机结合——贝叶斯优化算法的思想和实现细节。

\section{计算机实验设计}
虽然计算机实验设计与传统实验设计有着明显的差异,但两者还是有一些相似之处值得研究。
为了引入计算机实验设计的概念,本节将会先介绍传统实验设计的基本方法,
再介绍无模型的计算机实验设计方法,最后介绍统计模型计算机实验设计方法及其与全局最优化问题的联系。

\subsection{传统实验设计方法的演进}
实验设计(Design of Experiment)学科有着相当悠久的历史。
为了研究不同因素对系统的影响,科学界和工业界人士通常都会对其进行实验:
人为设置这些因素为特定的值,观察系统的输出结果,再对结果进行分析。
如何选定这些值以方便分析、如何进行分析以减弱不可控因素造成的影响的学问也就构成了实验设计学科\cite{davies1954}。
进行实验的目的一般有两类:判断某些因素的影响有无/强弱,和调整因素以提高系统性能。
前者意味着实验完毕进行数据分析时会采用线性回归分析和方差分析,
而后者意味着将会采用非线性回归分析或者更为复杂的手段,并经常意味着需要迭代多次实验。
本小节将先介绍设计基本原则,再分别讨论两类目标的设计手段。

\subsubsection{设计基本原则}
实验设计领域经典教材~\cite{montgomery}指出,为了对抗未知因素、提高实验结果的精度,(传统的)实验设计在进行设计时普遍遵循以下三大原则:
\begin{description}
  \item[随机化(Randomization)] 随机化可以将未知因素的影响控制在最小,以提高实验结果的可靠性。
  \item[重复实验(Replication)] 重复实验可以可以提高实验结果的精度,还可以得到实验误差的估计。
  \item[分块(Blocking)] 分块可以把不感兴趣的因素的影响和感兴趣的因素的影响抽离开,以得到正确的分析结果。
\end{description}

\subsubsection{判断某些因素的影响}
\begin{itemize}
  \item 偏重于:阶乘设计(Factorial Design)方法,又叫正交表\cite{刘瑞江2010}:
  \item 偏重于优化:
\end{itemize}
),

\subsubsection{优化}

正交实验设计
最优实验设计

\subsection{计算机实验设计的诞生}
计算机模型和实验源于对复杂自然现象的简化和模拟:通过建立计算机模型并求解,科学家可以避免进行复杂昂贵并且费时间的物理实验。
由于计算机模型具有确定性、没有噪声因素等显著特点\footnote{大部分对计算机模型的讨论都限于讨论确定性(deterministic)计算机模型,也即同一模型接受相同输入所得到的输出应该没有偏差且严格相同。本文也将遵循这一约定。},在设计计算机实验时的考虑将会和设计传统物理实验存在显著区别。\cite{sacks1989}


关于这一系列问题研究最早可以追溯到1979年。文献~\inlinecite{mckay1979}比较了两种传统实验设计方法——随机抽样和分层抽样,并提出了一种新的方法——Latin Hypercube抽样(简称Lh)。
为了在每个输入变量维度上都服从均匀分布,Lh方法先将每个输入变量的范围等分成$N$份,再从每份中随机抽取一点。
$N$组进行随机化。

\inlinecite{sacks1989}在总结了具体实际计算机实验设计问题

相较于传统实验设计通过随机化和分集的手段来控制随机误差和未知因素,
计算机实验设计\cite{mckay1979,sacks1989}
% TODO
\subsection{统计模型的引入}
% TODO
\section{全局最优化}
% TODO
\subsection{启发式全局优化算法}
% TODO
\subsection{非启发式全局优化算法}
% TODO
\subsection{统计模型的引入}
% TODO
\section{贝叶斯优化算法的实现细节}
% TODO
\subsection{统计模型}
% TODO
\subsection{收获函数}
% TODO
\subsection{约束条件}
% TODO
\subsection{并行性}
% TODO
\subsection{子优化问题的求解}
% TODO
\subsection{结论}
% TODO
\end{document}
