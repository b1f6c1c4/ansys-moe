\documentclass[index]{subfiles}
\begin{document}
\chapter{引言}\label{sec:intro}
\section{课题背景}
无线电能传输(Wireless Power Transmission)技术最早可以追溯到19世纪末尼古拉·特斯拉时期的系列研究\cite{barrett1894}。
这一技术对交通领域带来了一定的影响,有不少将无线电能传输技术应用在交通领域的初步的尝试\cite{eghtesadi1990}。
在2007年补偿网络方法由Solja\v{c}i\'c引入\cite{kurs2007}之后,基于补偿网络的磁谐振耦合式无线充电技术迅猛发展,
并以其传输距离长、传输功率大、传输效率高、成本低廉等诸多优势,在电动汽车领域取得了广泛的应用\cite{高大威2015}。

\begin{figure}[htb]
  \centering
  \includegraphics{./figures/dist/intro-arch.pdf}
  \caption[本文选题在电动汽车无线充电系统技术体系中的地位]{电动汽车无线充电系统技术体系。节选并改编自文献~\cite{高大威2015}。加粗带下划线部分为本文所重点研究的部分。\label{fig:intro-arch}}
\end{figure}

对磁谐振耦合式电动汽车无线充电系统来说,它是一个多物理系统耦合的复杂整体,与电力电子、电磁场等诸多学科互相交叉,需要大量的相关技术进行支撑。
文献~\inlinecite{高大威2015}从系统的本质属性和基本要求出发,讨论了电动汽车无线充电系统中存在的科学问题和相应的支撑技术。
为了提高系统效率,需要解决多源能量双向耦合管理的科学问题;其中关键的一项也就是对磁谐振部件(即传输线圈、铁芯等)进行电磁场分析、计算与仿真。
实际上,仅对电磁场进行仿真解算还远远不够,应该利用计算结果对这些部件的材料、形状、尺寸等参数进行合理地调整,以得到更高的效率。
这一过程一般称作优化设计。基于电磁场仿真的优化设计过程牵扯到有限元方法、计算机实验设计、全局最优化等数学工具,
且往往需要采用软件工程的方法,开发自动化设计软件来辅助设计者完成系统的优化设计。
\cref{fig:intro-arch}从电动汽车无线充电系统的本质属性和基本要求出发,完整交代了本文的选题在上述知识体系(只展开描绘了和本文有直接联系的部分)中的位置,
为后续分析描绘了基本图景。

\section{研究现状}
对于无线充电系统磁谐振部件的设计问题,现有文献已经给出了相当多的解决方案。
综述~\cite{张艺明2016}总结了几种适用于通用磁谐振式WPT系统的传输线圈和铁芯的结构。
而针对电动汽车这一特殊应用,磁谐振部件可以根据充电过程中汽车的运动状态分为动态与静态两大类\cite{赵争鸣2016}。
由于动态充电系统(发射端一般是通电轨道,接收端一般是汽车上的接收线圈)的物理结构和设计分析方法都相对复杂,本文将讨论局限在静态(也即固定式)充电系统。

\subsection{线圈结构}
固定式电动汽车无线充电系统的磁耦合系统按基本结构划分,可以分为平面式(亦称单边式、平板式)和螺线管式(亦称双边式、圆柱式)\cite{高大威2015}。
平面式的基本特点是(对发射端或者接收端而言)铁磁材料位于下方,传输线圈水平放在其上;
而螺线管式的基本特点则是铁磁材料位于中央,传输线圈缠绕在铁磁材料之上。

平面式磁耦合系统再做细分可以分为圆形\cite{budhia2011}、DD\cite{budhia2013}、DDQ\cite{budhia2013}、BP\cite{covic2011}、方形\cite{choi2014}等等。
文献~\inlinecite{zaheer2012,zaheer2015}对比了上述几种线圈结构并指出采用DDQ和BP作为接收线圈具有较好的充电效果。

关于螺线管式磁耦合系统,常见结构包括Flux pipe\cite{budhia2010}、H\cite{takanashi2012}等。
文献~\inlinecite{haldi2013}利用有限元分析的手段,对平面式磁耦合系统和螺线管式磁耦合系统进行了详细的比较,并指出平面式磁耦合系统相比于螺线管式有着更多的优势。

\subsection{优化设计}

在线圈结构已知,但具体尺寸尚未确定的情况下,目前已知绝大部分文献\cite{budhia2011,shijo2015,tang2015}都采用暴力搜索——遍历所有可行解——的方法来进行优化设计,而罕有从数学角度和工程角度研究优化问题本身。
文献~\inlinecite{deng2016}采用在可行域内沿某一固定方向按给定步长进行线性搜索的方式来优化其提出的三线圈无线充电系统;
虽然该方法从数学和工程角度对优化问题的求解做出了初等的尝试,但该方法显然不能找到全局最优设计,且在本质上与暴力搜索区别不大。

多目标
T. Yilmaz, N. Hasan, R. Zane and Z. Pantic, "Multi-Objective Optimization of Circular Magnetic Couplers for Wireless Power Transfer Applications," in IEEE Transactions on Magnetics, vol. 53, no. 8, pp. 1-12, Aug. 2017.  doi: 10.1109/TMAG.2017.2692218
S. Bandyopadhyay, V. Prasanth, L. R. Elizondo and P. Bauer, "Design considerations for a misalignment tolerant wireless inductive power system for electric vehicle (EV) charging," 2017 19th European Conference on Power Electronics and Applications (EPE'17 ECCE Europe), Warsaw, 2017, pp. P.1-P.10.  doi: 10.23919/EPE17ECCEEurope.2017.8099320
S. Bandyopadhyay, V. Prasanth, P. Bauer and J. A. Ferreira, "Multi-objective optimisation of a 1-kW wireless IPT systems for charging of electric vehicles," 2016 IEEE Transportation Electrification Conference and Expo (ITEC), Dearborn, MI, 2016, pp. 1-7.  doi: 10.1109/ITEC.2016.7520210

\iffalse
6: budhia2010%
7: takanashi2012%
8: budhia2011%
9: budhia2013%
10: covic2011%
11: choi2014%
bgo4mppt abdelrahman2016
26: budhia2013%
92: zhang2014%WTF
93: budhia2011%
94: budhia2010%
95: takanashi2012%
96: nguyen2014%wtf
97: zaheer2015%
98: haldi2013%%
\fi%  
% TODO
\section{论文结构}
% TODO
\end{document}
