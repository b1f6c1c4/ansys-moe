\documentclass[index]{subfiles}
\begin{document}
\chapter{引言}\label{sec:intro}
\section{课题背景}
无线电能传输(Wireless Power Transmission)技术最早可以追溯到19世纪末尼古拉·特斯拉时期的系列研究\cite{barrett1894}。
这一技术对交通领域带来了一定的影响,有不少将无线电能传输技术应用在交通领域的初步的尝试\cite{eghtesadi1990}。
在2007年补偿网络方法由Solja\v{c}i\'c引入\cite{kurs2007}之后,基于补偿网络的磁谐振耦合式无线充电技术迅猛发展,
并以其传输距离长、传输功率大、传输效率高、成本低廉等诸多优势,在电动汽车领域取得了广泛的应用\cite{高大威2015}。

\begin{figure}[htb]
  \centering
  \includegraphics{./figures/dist/intro-arch.pdf}
  \caption[本文选题在电动汽车无线充电系统技术体系中的地位]{电动汽车无线充电系统技术体系。节选并改编自文献~\cite{高大威2015}。加粗带下划线部分为本文所重点研究的部分。\label{fig:intro-arch}}
\end{figure}

对磁谐振耦合式电动汽车无线充电系统来说,它是一个多物理系统耦合的复杂整体,与电力电子、电磁场等诸多学科互相交叉,需要大量的相关技术进行支撑。
文献~\inlinecite{高大威2015}从系统的本质属性和基本要求出发,讨论了电动汽车无线充电系统中存在的科学问题和相应的支撑技术。
为了提高系统效率,需要解决多源能量双向耦合管理的科学问题;其中关键的一项也就是对磁谐振部件(即传输线圈、铁芯等)进行电磁场分析、计算与仿真。
实际上,仅对电磁场进行仿真解算还远远不够,应该利用计算结果对这些部件的材料、形状、尺寸等参数进行合理地调整,以得到更高的效率。
这一过程一般称作优化设计。基于电磁场仿真的优化设计过程牵扯到有限元方法、计算机实验设计、全局最优化等数学工具,
且往往需要采用软件工程的方法,开发自动化设计软件来辅助设计者完成系统的优化设计。
\cref{fig:intro-arch}从电动汽车无线充电系统的本质属性和基本要求出发,完整交代了本文的选题在上述知识体系(只展开描绘了和本文有直接联系的部分)中的位置,
为后续分析描绘了基本图景。

\section{研究现状}
现有文献针对电动汽车无线充电系统的

6: budhia2010
7: takanashi2012
8: budhia2011
9: budhia2013
10: covic2011
11: 
\iffalse
bgo4mppt abdelrahman2016
15: kurs2007
26: budhia2013
92: zhang2014
93: budhia2011
94: budhia2010
95: takanashi2012
96: nguyen2014
97: zaheer2015
98: haldi2013
\fi%  
% TODO
\section{论文结构}
% TODO
\end{document}
