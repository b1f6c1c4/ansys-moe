\documentclass[index]{subfiles}
\begin{document}
\chapter{自动化设计系统的实现}\label{sec:impl}
本章将在\cref{sec:design}对系统的顶层设计的基础上,重点讨论每个组件的具体设计与实现细节,并介绍具体软件实现中所采用的技术栈。

对于较为大型的软件系统,除了开发系统本身以外,还需要单独开发另一套系统,用以对实际系统进行时时刻刻的监控,以方便运维人员应对可能出现的异常情况。
本章将先介绍自动化设计系统的本身(应用面,面向用户)的实现细节,再简要介绍内部监控系统(控制面,并不面向用户)的架构,
最后采用UML部署图画出整个系统的物理视图,讨论各个组件之间的通信方法,并描述整个系统的最终实现成果。

\section{应用面}
本节将就\cref{fig:design-comp}中的每个组件分别讨论技术栈选型和设计实现的全部细节。

在\cref{fig:design-comp}中,状态存储和消息队列两个组件和其他组件存在显著不同——它们都已经有了非常成熟的开源软件实现。
对于状态存储,本文选用开源分布式强一致性键值数据库etcd作为实现,其对外提供了读取、写入、擦除、监控修改等等一系列功能,
可以满足\cref{fig:design-comp}中对状态存储组件的要求。

对于消息队列,本文选用基于Erlang语言的开源分布式消息队列RabbitMQ作为实现,其遵守AMQP协议,且对外提供了队列运行监控功能。
通过将其内部结构配置成\cref{fig:design-comp-mq}所示的样子,就可以满足\cref{fig:design-comp}中对消息队列组件的要求。

\subsection{文件存储}
\begin{figure}[h]
  \centering
  \includegraphics{./figures/dist/impl-storage-usecase.pdf}
  \caption{文件存储组件的行为:UML用户用例图\label{fig:impl-storage-usecase}}
\end{figure}
\begin{figure}[h]
  \centering
  \includegraphics{./figures/dist/impl-storage-package.pdf}
  \caption{文件存储组件的结构:UML包图\label{fig:impl-storage-package}}
\end{figure}
通过将\cref{fig:design-comp}中对文件存储组件的要求细化,可以得到\cref{fig:impl-storage-usecase}。
具体来说,文件存储组件对外提供一套HTTP API,根据HTTP动词和URL来对硬盘上的指定资源进行指定操作,如上传、下载、移动等等。
之所以要区分两种不同的上传,是因为需要对用户上传的仿真源文件进行重命名,使得文件名相同的文件内容相同,文件名不同的文件文件内容不同,
以保证\cref{sec:design-wf}中提到的对Ansys结果的缓存可以正确工作。

本文选取JavaScript语言进行程序编写,文件结构如\cref{fig:impl-storage-package}所示,其中主要文件的功能如下:
\begin{description}
  \item[logger.js] 将日志信息汇总至Logstash(见\cref{sec:impl-elk})
  \item[file/common.js] 判断文件名是否合法
  \item[file/get.js] 响应GET请求,实现文件下载、列出文件夹内容、打包下载文件夹
  \item[file/post.js] 响应POST请求,实现源文件上传、仿真结果上传
  \item[file/put.js] 响应PUT请求,方便调试时上传文件
  \item[utils/contentstream.js] 对HTTP PUT请求的内容进行解析
  \item[utils/disk.js] 对用户上传的仿真源文件进行重命名
\end{description}

\subsection{工作流内核}
% TODO
\subsection{计算服务}
% TODO
\subsection{网站后端}
% TODO
\subsection{网站前端}
% TODO
\subsection{前置代理}
% TODO

\section{控制面}\label{sec:impl-elk}
% TODO
\section{物理视图}
\begin{figure}[h]
  \centering
  \includegraphics{./figures/dist/impl-deploy.pdf}
  \caption{系统物理视图:UML部署图\label{fig:impl-deploy}}
\end{figure}
% TODO
\end{document}
