\documentclass[index]{subfiles}
\begin{document}
\chapter{自动化设计系统的设计}\label{sec:design}
在\cref{sec:doebgo}中已经对贝叶斯优化算法的诸多方面的细节问题进行了讨论。
然而,从算法本身到软件系统的过程又是一个相当不平凡的过程。
本章和下一章将紧扣软件实现这一主题,对自动化设计系统进行全方位的建模与设计。

经过上个世纪80至90年代软件工程领域的蓬勃发展,“4+1”视图模型\cite{kruchten1995}是在上个世纪90年代中叶被提出。
其主张通过以下5个视图(View)来对复杂软件系统进行建模:
\begin{description}
  \item[逻辑(Logical)视图] 描述系统给用户提供何种功能
  \item[流程(Process)视图] 描述工作流中的并发和同步
  \item[开发(Development)视图] 描述开发过程中软件的静态组织关系
  \item[物理(Physical)视图] 描述软件与硬件的映射关系
  \item[场景(Scenario)视图] 基于某个用户用例(场景),展示以上四个视图如何配合
\end{description}

在该理论提出之初,各个视图普遍采用当时最为流行的Booch记号进行绘制,且一切基于当时非常看好的面向对象编程范式。
虽然软件工程领域在其后二十年内沧海桑田,Booch记号早已淘汰,UML语言以其强大的表现力和规范化成为了软件模型的绝对标准,面向函数范式渐渐取代面向对象,
但“4+1”视图的基本原理依然沿用至今,只不过普遍采用UML语言进行表达。

本文考虑到自动化设计系统软件的特殊性——交互较少(提交任务以后基本无需干预),但工作流却异常复杂(执行非常复杂的算法),
在“4+1”视图的基本思想指导下,选取其中部分视图对系统进行建模:
\begin{description}
  \item[逻辑视图] 采用UML用户用例图来描述系统给用户提供何种功能(\cref{sec:design-usecase})
  \item[(优化场景下的)流程视图] 采用UML活动图和FL-Petri网描述优化工作流中的并发和同步(\cref{sec:design-wf})
  \item[开发视图] 采用UML组件图描述开发过程中软件的静态组织关系(\cref{sec:design-comp};\cref{sec:impl}中会对其进行细化)
  \item[物理视图] 采用UML部署图描述软件与硬件的映射关系(将在\cref{sec:impl}中讨论)
\end{description}

\section{逻辑视图:用户用例}\label{sec:design-usecase}
\begin{figure}[h]
  \centering
  \includegraphics{./figures/dist/design-usecase.pdf}
  \caption{系统逻辑视图:UML用户用例图\label{fig:design-usecase}}
\end{figure}
自动化设计系统只有设计者一类用户。\Cref{fig:design-usecase}具体描述了系统可以给设计者提供的功能(用户用例)。

\section{流程视图:贝叶斯优化算法的工作流建模}\label{sec:design-wf}
\begin{figure}[h]
  \centering
  \includegraphics{./figures/dist/design-activity.pdf}
  \caption{系统流程视图:UML活动图\label{fig:design-activity}}
\end{figure}
本节

\section{开发视图:系统核心组件划分}\label{sec:design-comp}
% TODO
\end{document}
