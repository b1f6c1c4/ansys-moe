\documentclass[degree=bachelor,tocarialchapter]{../thuthesis/thuthesis}
\usepackage{./thuthesis/thuthesis}
\usepackage{./thuappendixbib/thuappendixbib}
\usepackage{subfiles}
\usepackage{siunitx}
\usepackage{cleveref}
\usepackage{upgreek}
\usepackage{listings}
\usepackage{mathtools}
\thusetup{%
  ctitle={电动汽车无线充电系统\\磁谐振部件优化设计软件研究},
  etitle={Software research for optimal design of magnetic resonant components of EV wireless charging system},
  cauthor={涂金正},
  eauthor={Tu, Jinzheng},
  cdegree={工学学士},
  edegree={Bachelor of Engineering},
  cdepartment={电机工程与应用电子技术系},
  edepartment={Department of Electrical Engineering},
  cmajor={电气工程及其自动化},
  emajor={Electrical Engineering},
  csupervisor={陈凯楠\quad{}助理研究员},
  esupervisor={Chen, Kainan Assistant Professor},
  cdate={2018年06月10日},
  edate={2018-06-10},
}
\sisetup{%
  binary-units=true,list-separator={, },
  list-final-separator={, },
  list-pair-separator={, },
  separate-uncertainty,
  range-phrase={\ensuremath{\sim}},
  math-micro=\upmu,
  text-micro=\upmu,
}
\lstset{%
  basicstyle=\ttfamily\small,
  extendedchars=false,
}
\newcommand\crefpairgroupconjunction{和}\newcommand\crefmiddlegroupconjunction{、}\newcommand\creflastgroupconjunction{和}
\newcommand\crefrangeconjunction{至}\newcommand\crefrangepreconjunction{}\newcommand\crefrangepostconjunction{}
\newcommand\crefpairconjunction{和}\newcommand\crefmiddleconjunction{、}\newcommand\creflastconjunction{和}
\crefformat{chapter}{第~#1~章}\Crefformat{chapter}{第~#1~章}
\crefrangeformat{chapter}{第~#1~至~#2~章}\Crefrangeformat{chapter}{第~#1~至~#2~章}
\crefformat{appendix}{附录~#1}\Crefformat{appendix}{附录~#1}
\crefformat{listing}{程序~#1}\Crefformat{listing}{程序~#1}
\crefformat{equation}{式~#1}\Crefformat{equation}{式~#1}
\crefformat{section}{节~#1}\Crefformat{section}{节~#1}
\crefformat{subappendix}{节~#1}\Crefformat{subappendix}{节~#1}
\crefformat{subsubappendix}{节~#1}\Crefformat{subsubappendix}{节~#1}
\crefformat{subsubsubappendix}{节~#1}\Crefformat{subsubsubappendix}{节~#1}
\crefformat{figure}{图~#1}\Crefformat{figure}{图~#1}
\crefformat{table}{表~#1}\Crefformat{table}{表~#1}
\crefformat{definition}{定义~#1}\Crefformat{definition}{定义~#1}
\AtBeginDocument{% ...if you're using hyperref
  \let\oldlabel\label% Copy original version of \label
  \let\oldref\ref% Copy original version of \ref
  \let\oldcref\cref% Copy original version of \cref
  \let\oldCref\Cref% Copy original version of \Cref
}
\newcommand{\addlabelprefix}[1]{%
  \renewcommand{\label}[1]{\oldlabel{#1-##1}}% Update \label
  \renewcommand{\ref}[1]{\oldref{#1-##1}}% Update \ref
}
\newcommand{\addlabelprefixc}[1]{%
  \renewcommand{\label}[1]{\oldlabel{#1-##1}}% Update \label
  \renewcommand{\cref}[1]{\oldcref{#1-##1}}% Update \cref
  \renewcommand{\Cref}[1]{\oldCref{#1-##1}}% Update \Cref
}
\newcommand{\removelabelprefix}{%
  \renewcommand{\label}{\oldlabel}% Restore \label
  \renewcommand{\ref}{\oldref}% Restore \ref
  \renewcommand{\cref}{\oldcref}% Restore \cref
  \renewcommand{\Cref}{\oldCref}% Restore \Cref
}
% For subfiles
\makeatletter%
\@ifclassloaded{subfiles}%
  {%\AtBeginDocument{\Denotation}
  \AtEndDocument{%
    \bibliographystyle{thuthesis-author-year}
    \bibliography{index}
  }}%
  {\relax}%
\makeatother%
\newcommand{\one}{{1}}
\let\oldPr\Pr
\renewcommand{\Pr}[1]{\oldPr\!\left[#1\right]}
\newcommand{\E}[1]{\operatorname{E}\!\left[#1\right]}
\newcommand{\Var}[1]{\operatorname{Var}\!\left[#1\right]}
\newcommand{\Cov}[2]{\operatorname{Cov}\!\left[#1,#2\right]}
\newcommand{\Corr}[2]{\operatorname{Corr}\!\left[#1,#2\right]}
\newcommand{\Normal}[2]{\operatorname{N}\!\left(#1,#2\right)}
\DeclareMathOperator*{\argmin}{argmin}
\DeclareMathOperator*{\argmax}{argmax}
\makeatletter
\renewcommand{\thu@denotation@name}{主要符号表}
\ifdefined\stripped
\renewcommand{\thu@authorization@mk}{\relax}
\renewcommand{\endacknowledgement}{\relax}
\fi
\makeatother
\newcommand{\Denotation}[0]{%
\begin{denotation}
  \item[$\one_n$] $n\times{}1$的全1列向量
  \item[$\Pr{\cdot}$] 某事件发生的概率
  \item[$\E{\cdot}$] 随机变量的期望
  \item[$\Var{\cdot}$] 随机变量的方差
  \item[$\Cov{\cdot}{\cdot}$] 两个随机变量的协方差
  \item[$\Corr{\cdot}{\cdot}$] 两个随机变量的相关系数
  \item[$\beta$] 随机过程的均值
  \item[$\sigma^2$] 随机过程的方差
  \item[$\Normal{\mu}{\Sigma}$] (多元)正态分布
  \item[$n$] 已完成实验的数目
  \item[$x_s\quad x_i$] 已完成实验的位置(集合和元素)
  \item[$y_s\quad y_i$] 已完成实验的结果(集合和元素)
  \item[$p$] 进行中实验的数目
  \item[$x_\ast\quad x_{\ast i}$] 进行中实验的位置(集合和元素)
  \item[$x_\star$] 下一步应该进行的实验的位置
  \item[$Y(x)$] $x$处的实验结果对应的随机变量(先验)
  \item[$Y(x)|y_s$] $x$处的实验结果对应的随机变量(后验)
  \item[$K$] 已完成实验处对已完成实验处随机变量的归一化协方差矩阵
  \item[$K_s$] 未完成实验处\footnote{包括进行中实验和下一步应该进行的实验,下同}对已完成实验处随机变量的归一化协方差矩阵
  \item[$K_{ss}$] 未完成实验处对未完成实验处随机变量的归一化协方差矩阵
\end{denotation}
}
\setcounter{tocdepth}{2}
\newcommand{\ltnum}[1]{\num[%
  parse-numbers=true,%
  round-mode=figures,%
  round-integer-to-decimal=true,
  round-precision=4%
]{#1}}
\newcommand{\tnum}[1]{\num[%
  parse-numbers=true,%
  round-mode=figures,%
  round-integer-to-decimal=true,
  round-precision=3%
]{#1}}
\def\exmHour{\SI{24}{\hour}}
\input{./data/dist/opt.tex}

\begin{document}
\frontmatter

\begin{cabstract}
  用于电动汽车的无线电能传输(WPT)技术在近十年间吸引了很多关注。
  针对其中磁谐振部件的电磁场仿真及优化设计问题,现有文献所采用的方法需要相对较多次的迭代和电磁场仿真才能找到全局最优设计方案,
  也没有针对无线充电系统设计问题本身的特点进行优化。
  本文从分析一般性电动汽车无线充电系统磁谐振部件优化设计问题本身的特点出发,展开文献研究,选取最适合该问题的贝叶斯优化算法作为内核,
  利用软件工程的技术手段,开发了计算机自动化设计软件系统,实现了基于该优化算法的WPT磁谐振部件优化设计功能。
  针对5个设计变量、6个设计目标、数百万种设计可能性的优化算例表明,该自动化设计系统对搜索位置的选取合理,比暴力搜索和完全随机搜索更省资源;
  该系统还实现了无人值守自动运行,将错误控制在最小范围内,问题解决后可以从上次中断位置无缝继续运行。
\end{cabstract}
\ckeywords{无线充电, 磁耦合, 优化设计, 自动化设计}

\begin{eabstract}
  Wireless Power Transfer (WPT) for EV charging has drawn much attraction over the past decades. Regarding electromagnetic field analysis and optimal design problem of magnetic resonant components, existing literature mostly applied inefficient algorithms that require a large number of iteration to find the global optima. Furthermore, none of them tunes the algorithms according to the unique characteristics of WPT optimal design problem. By analyzing the features of the design problem ahead of time, we choose the most suitable algorithm, Bayesian Optimization, from related literature. Utilizing software engineering methodologies, we develop a computer-automated design (CautoD) software system that fulfills the need of WPT magnetic resonant component optimal design with such optimization algorithm. A case study involving five design variables, six optimization targets and millions of design possibilities indicates that the CautoD system appropriately chooses sampling points, saving a considerable amount of resources than brute-force or random searching. Moreover, the system permits unattended operating, confines errors to the minimum scope, and allows seamless recovering from exceptions.
\end{eabstract}
\ekeywords{Wireless Power Transfer, Inductive Coupling, Optimal Design, Computer-automated Design}

\makecover

\tableofcontents

\Denotation

\mainmatter

\subfile{introduction}
\subfile{features}
\subfile{doebgo}
\subfile{petri}
\subfile{design}
\subfile{impl}
\subfile{example}
\subfile{conclusion}

\backmatter

\listoffigures
\listoftables
\listofequations

\bibliographystyle{thuthesis-numeric}
\bibliography{index}

\begin{acknowledgement}
  感谢导师赵争鸣教授、陈凯楠老师对本研究的提纲挈领的指导。

  感谢Scott Clark的博士论文\cite{clark2012}及其相应的开源软件项目MOE——EPI算法的C语言实现,为本文的R语言实现提供了重大帮助。

  在美国弗吉尼亚理工大学进行三个月的研究期间,承蒙李泽元教授、郦强教授、冯君杰师兄的热心指导与帮助,不胜感激。

  感谢刘昊天同学在系统需求分析、硬件设备、论文排版上的帮助。

  感谢在软件开发和论文写作过程中的所有开源库和工具链
  (涵盖JavaScript、Golang、R、\LaTeX、Python、VBScript、Makefile、Java、C、Erlang、Ruby、Mathematica、Perl等多门语言的数千项目和软件包,
  除了在算法上直接用到的部分R和Python包在参考文献中引用了以外,恕不能一一列出)。
  尤其是\thuthesis\footnote{\href{https://github.com/xueruini/thuthesis}{https://github.com/xueruini/thuthesis}},帮我节省了不少精力。
\end{acknowledgement}

\begin{appendix}
\ifdefined\stripped\relax\else
\addtocontents{toc}{\protect\setcounter{tocdepth}{0}}
\addtocontents{lof}{\protect\iffalse}
\addtocontents{lot}{\protect\iffalse}
\addtocontents{loe}{\protect\iffalse}
\addlabelprefixc{trans}
\subfile{trans}
\removelabelprefix
\addlabelprefix{raw}
\subfile{raw}
\removelabelprefix
\addtocontents{toc}{\protect\setcounter{tocdepth}{5}}
\addtocontents{lof}{\protect\fi}
\addtocontents{lot}{\protect\fi}
\addtocontents{loe}{\protect\fi}
\fi
\subfile{tricks}
\end{appendix}

\end{document}
